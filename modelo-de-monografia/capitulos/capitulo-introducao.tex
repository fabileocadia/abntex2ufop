% ----------------------------------------------------------
% Introdução (exemplo de capítulo sem numeração, mas presente no Sumário)
% ----------------------------------------------------------
\chapter[Introdução]{Introdução}
%\addcontentsline{toc}{chapter}{Introdução}
% O Forno de Indução pode ser utilizado em indústrias metalúrgicas, possibilitando a fusão de metais, produção de ligas metálicas, etc. Descoberto por Michael Faraday, a indução começa com uma bobina de material condutor (por exemplo, cobre). Sempre que um campo magnético que atravessa uma espira variar, aparece nesse circuito uma corrente elétrica, Figura 1. Esse fenômeno é chamado de indução eletromagnética.
%Equipamentos de aquecimento por indução requerem uma compreensão da física, eletromagnetismo, eletrônica de potência e controle de processos, mas os conceitos básicos por trás de aquecimento por indução são simples de entender.
%Produtos aquecidos por indução não dependem de convecção ou radiação para aquecer uma peça. Em vez disso, o aquecimento é gerado na superfície da peça pelo fluxo da corrente e então transferido para o núcleo através da condução térmica.
%A profundidade de aquecimento depende muito da frequência da corrente alternada que flui através da peça obra. Uma maior frequência da corrente resultará em menor profundidade de aquecimento e a menor frequência irá resultar em uma maior profundidade de aquecimento. Essa profundidade também dependerá das propriedades elétricas e magnéticas da peça obra.
%O Forno de Indução é um transformador, com 250 espiras na bobina primária e apenas uma espira na bobina secundária. Esta espira na bobina secundária é na verdade um anel metálico de alumínio com espaço onde é possível armazenar água. A redução de 1/250 na tensão de entrada, de 220 volts para pouco mais de 0,8 volts faz com que, em compensação, a corrente no anel seja muito alta, permitindo que uma grande potência elétrica seja dissipada no anel. Essa potência é dissipada em forma de calor (efeito Joule), esquentando o anel e fazendo a água armazenada nele ferver em poucos segundos.
%Dentre as principais vantagens podemos citar a limpeza, a menor contaminção do material a ser fundido/tratado, o custo do consumo de energia (este obviamente, depende da política energética de cada país), etc; As desvantagens são, principalmente o custo do equipamento, e também a periculosidade do equipamento, já que este deve ser operado por pessoa melhor qualificada, sendo que neste equipamento é necessário conhecimento do que deve ser feito em caso de falta de energia por exemplo, o operador deve saber que está trabalhando com equipamento que manuseia altas tensões e correntes, etc.





\section{Objetivos}
%Projetar um protótipo forno de indução para fusão de metais, levando em consideração a base de estudo da disciplina de Eletrotécnica
\end{itemize}

\lipsum[7]

\lipsum[8]

\section{Organiza{\c c}{\~a}o e estrutura}

\lipsum*[9-11]

\begin{itemize}
\item item;
\item item;
\item item;
\item item;
\end{itemize}

\section{Cronograma}

Esta seção deve constar somente no projeto de monografia. Não deve aparecer na versão final do texto.

% Please add the following required packages to your document preamble:
% \usepackage[table,xcdraw]{xcolor}
% If you use beamer only pass "xcolor=table" option, i.e. \documentclass[xcolor=table]{beamer}

% Please add the following required packages to your document preamble:
% \usepackage[table,xcdraw]{xcolor}
% If you use beamer only pass "xcolor=table" option, i.e. \documentclass[xcolor=table]{beamer}

Um exemplo de cronograma das atividades é proposto na tabela \ref{tab:cronograma}.\footnote{Voc{\^e} pode elaborar também tabelas online, gerando o código em \LaTeX. Após isso, basta copiar e colar o código aqui. Um exemplo de site é o ``Table Generator''\url{http://www.tablesgenerator.com/}.}
  
\begin{table}[h]
\ABNTEXfontereduzida
\caption[Cronograma das atividades]{Cronograma das atividades de elaboração da monografia.}
\label{tab:cronograma}
\begin{minipage}{0.3\textwidth}
    \centering
\begin{tabular}{|l|l|l|l|l|l|l|l|l|l|l|l|l|l|l|l|l|}
\hline
                             & \multicolumn{16}{c|}{Meses}                                                   \\ \hline
Atividades (Etapas)          & 01 & 02 & 03 & 04 & 05 & 06 & 07 & 08 & 09 & 10 & 11 & 12 & 13 & 14 & 15 & 16 \\ \hline
1. Estudo da teoria          & X  & X  & X  & X  & X  &    &    &    &    &    &    &    &    &    &    &    \\ \hline
2. Atualização bibliográfica & X  & X  & X  & X  & X  & X  & X  &    &    &    &    &    &    &    &    &    \\ \hline
3. Seleção de Material       & X  & X  & X  & X  & X  & X  & X  & X  & X  &    &    &    &    &    &    &    \\ \hline
4. Elaboração da monografia  &    &    &    &    & X  & X  & X  & X  & X  & X  & X  & X  & X  & X  & X  &    \\ \hline
5. Elaboração de Artigo      &    &    &    &    &    &    &    &    & X  & X  & X  & X  & X  & X  & X  &    \\ \hline
6. Defesa da monografia      &    &    &    &    &    &    &    &    &    &    &    &    &    &    &    & X  \\ \hline
\end{tabular}
  \end{minipage}
\end{table}
