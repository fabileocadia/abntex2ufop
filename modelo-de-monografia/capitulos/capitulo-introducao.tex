
% Introdução (exemplo de capítulo sem numeração, mas presente no Sumário)
% ----------------------------------------------------------
\chapter[Introdução]{Introdução}
%\addcontentsline{toc}{chapter}{Introdução}
O Forno de Indução pode ser utilizado em indústrias metalúrgicas, possibilitando a fusão de metais, produção de ligas metálicas, etc. Descoberto por Michael Faraday, a indução começa com uma bobina de material condutor (por exemplo, cobre). Sempre que um campo magnético que atravessa uma espira variar, aparece nesse circuito uma corrente elétrica, Figura 1. Esse fenômeno é chamado de indução eletromagnética.
\section{Justificativas e Relev{\^a}ncia}
%
Este documento e seu código-fonte são exemplos de referência de uso da classe
\textsf{abntex2} e do pacote \textsf{abntex2cite}. O documento 
exemplifica a elaboração de trabalho acadêmico (tese, dissertação e outros do
gênero) produzido conforme a ABNT NBR 14724:2011 \emph{Informação e documentação
- Trabalhos acadêmicos - Apresentação}.
A expressão ``Modelo Canônico'' é utilizada para indicar que \abnTeX\ não é
modelo específico de nenhuma universidade ou instituição, mas que implementa tão
somente os requisitos das normas da ABNT. Uma lista completa das normas
observadas pelo \abnTeX\ é apresentada em \citeonline{abntex2classe}.

Sinta-se convidado a participar do projeto \abnTeX! Acesse o site do projeto em
\url{http://abntex2.googlecode.com/}. Também fique livre para conhecer,
estudar, alterar e redistribuir o trabalho do \abnTeX, desde que os arquivos
modificados tenham seus nomes alterados e que os créditos sejam dados aos
autores originais, nos termos da ``The \LaTeX\ Project Public
License''\footnote{\url{http://www.latex-project.org/lppl.txt}}.

Encorajamos que sejam realizadas customizações específicas deste exemplo para
universidades e outras instituições --- como capas, folha de aprovação, etc.
Porém, recomendamos que ao invés de se alterar diretamente os arquivos do
\abnTeX, distribua-se arquivos com as respectivas customizações.
Isso permite que futuras versões do \abnTeX~não se tornem automaticamente
incompatíveis com as customizações promovidas. Consulte
\citeonline{abntex2-wiki-como-customizar} par mais informações.

Este documento deve ser utilizado como complemento dos manuais do \abnTeX\ 
\cite{abntex2classe,abntex2cite,abntex2cite-alf} e da classe \textsf{memoir}
\cite{memoir}. 

Esperamos, sinceramente, que o \abnTeX\ aprimore a qualidade do trabalho que
você produzirá, de modo que o principal esforço seja concentrado no principal:
na contribuição científica.

Equipe \abnTeX 

Lauro César Araujo
%
\section{Metodologia}

\lipsum[1]

\section{Objetivos}

\lipsum[7]

\lipsum[8]

\section{Organiza{\c c}{\~a}o e estrutura}

\lipsum*[9-11]

\begin{itemize}
\item item;
\item item;
\item item;
\item item;
\end{itemize}

\section{Cronograma}

Esta seção deve constar somente no projeto de monografia. Não deve aparecer na versão final do texto.

% Please add the following required packages to your document preamble:
% \usepackage[table,xcdraw]{xcolor}
% If you use beamer only pass "xcolor=table" option, i.e. \documentclass[xcolor=table]{beamer}

% Please add the following required packages to your document preamble:
% \usepackage[table,xcdraw]{xcolor}
% If you use beamer only pass "xcolor=table" option, i.e. \documentclass[xcolor=table]{beamer}

Um exemplo de cronograma das atividades é proposto na tabela \ref{tab:cronograma}.\footnote{Voc{\^e} pode elaborar também tabelas online, gerando o código em \LaTeX. Após isso, basta copiar e colar o código aqui. Um exemplo de site é o ``Table Generator''\url{http://www.tablesgenerator.com/}.}
  
\begin{table}[h]
\ABNTEXfontereduzida
\caption[Cronograma das atividades]{Cronograma das atividades de elaboração da monografia.}
\label{tab:cronograma}
\begin{minipage}{0.3\textwidth}
    \centering
\begin{tabular}{|l|l|l|l|l|l|l|l|l|l|l|l|l|l|l|l|l|}
\hline
                             & \multicolumn{16}{c|}{Meses}                                                   \\ \hline
Atividades (Etapas)          & 01 & 02 & 03 & 04 & 05 & 06 & 07 & 08 & 09 & 10 & 11 & 12 & 13 & 14 & 15 & 16 \\ \hline
1. Estudo da teoria          & X  & X  & X  & X  & X  &    &    &    &    &    &    &    &    &    &    &    \\ \hline
2. Atualização bibliográfica & X  & X  & X  & X  & X  & X  & X  &    &    &    &    &    &    &    &    &    \\ \hline
3. Seleção de Material       & X  & X  & X  & X  & X  & X  & X  & X  & X  &    &    &    &    &    &    &    \\ \hline
4. Elaboração da monografia  &    &    &    &    & X  & X  & X  & X  & X  & X  & X  & X  & X  & X  & X  &    \\ \hline
5. Elaboração de Artigo      &    &    &    &    &    &    &    &    & X  & X  & X  & X  & X  & X  & X  &    \\ \hline
6. Defesa da monografia      &    &    &    &    &    &    &    &    &    &    &    &    &    &    &    & X  \\ \hline
\end{tabular}
  \end{minipage}
\end{table}
